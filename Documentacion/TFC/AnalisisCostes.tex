Como se ha comentado anteriormente, el trabajo desarrollado ha sido una demo jugable de lo que, de continuarse el desarrollo se convertiría en un juego completo.



Debido a esto, se marcó el objetivo de mantener el coste del desarrollo sobre mínimos. Para conseguir cumplir con este objetivo se han llevado a cabo las siguientes acciones:
\begin{itemize}
	\item Usar assets libres o de uso gratis de sitios como el \textit{Unreal Marketplace} o \textit{BlendSwap}.
	\item Usar herramientas colaborativas gratuitas como \textit{GitHub} o \textit{Trello}.
	\item El uso de un motor de juegos gratuito y altamente integrado con el flujo de trabajo que permita no depender de herramientas externas.
\end{itemize}

El desarrollo del proyecto empezó el 27 de mayo teniendo como fecha de finalización el 24 de septiembre, una duración de 121 días de los cuales la dedicación media al proyecto ha sido de 2 h al día. 
Según las últimas encuestas del INE \todo{Insertar Cita} en la que sitúan el sueldo medio de los programadores informáticos en 19,7 \euro/h por lo que el coste de cada programador sería de 4767 \euro.

Teniendo en cuenta todo lo expuesto anteriormente el coste del proyecto ascendería a 14301 \euro

\begin{table}[!h]
\centering
\begin{tabular}{cccc}
\hline
\textbf{Concepto} & \textbf{Unidades} & \textbf{Coste Unitario} & \textbf{Coste Total} \\ \hline
Assets            & 0                 &                         & 0 \euro                  \\ \hline
Herramientas      & 0                 &                         & 0 \euro                  \\ \hline
Motor de Juegos   & 0                 &                         & 0 \euro                  \\ \hline
Salario personal  & 3                 & 4767 \euro                    & 14301 \euro               \\ \hline
\textbf{Coste final}       &                   &                         & \textbf{14301 \euro}               \\ \hline
\end{tabular}
\caption{}
\end{table} \todo{Añadir referencia de la tabla}