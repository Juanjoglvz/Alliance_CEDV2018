\lettrine[lines=2,findent=2pt,nindent=3pt,loversize=0.1]{\textcolor[gray]{0.4}{E}}{n} este capítulo se van a presentar los objetivos definidos para este proyecto que se encuentra en el ámbito del \ac{CEDV}. Se va a presentar el objetivo general del proyecto, acompañado de los objetivos específicos del mismo.

\subsection{Objetivo general}

El objetivo principal del proyecto será el diseño y desarrollo de un videojuego multijugador completo. Se llevarán a cabo todas las fases, desde la concepción (que se inicia con la idea del videojuego que se quiere crear), la creación de la historia y personajes, diseño de niveles y por último la implementación en un lenguaje de programación y las pruebas.

\subsection{Objetivos específicos}

Para llevar a cabo el objetivo general será necesario completar cada uno de los siguientes objetivos específicos:

\subsubsection{Subsistema Online}

Para conseguir que la campaña del juego pueda ser disfrutada por dos jugadores en distintos dispositivos, se va a llevar a cabo la integración del videojuego con el Subsistema Online de STEAM. Como resultado, se podrá jugar al videojuego a través de Internet, sin limitación geográfica.

\subsubsection{Diseño de Escenarios y Mecánicas de juego}

Es necesario conseguir unos escenarios atractivos visualmente y acordes a la historia para alcanzar la mayor inmersión posible en el juego por parte del jugador. De igual forma, las mecánicas de juego deben coincidir con el tipo de juego planteado y la historia del mismo, para dar el mayor sentido posible al videojuego.

\subsubsection{Diseño e Implementación del Minijuego}

Para añadir una jugabilidad extra que dote al juego de más contenido y permita al jugador relajarse, se introducirá en el juego una forma para acceder a ciertos lugares, basada en la resolución de un puzzle. 

\subsubsection{Personajes del videojuego e \ac{IA}}

Se diseñarán y codificarán todos los personajes que van a aparecer en el juego, tanto aliados como enemigos. Se va a diseñar la \ac{IA} correspondiente de cada uno, dotándoles de un comportamiento aparentemente inteligente para ofrecer una mayor sensación de inmersión al jugador. Los personajes serán capaces de identificar al jugador, moverse hacia él y atacarle, si se encuentran a una distancia propicia para ello.

\subsubsection{Diseño de la \ac{GUI}}

Se procederá a realizar un diseño de la interfaz gráfica del juego, interfaz de diálogos y distintos menús que conforman \textit{Alliance}, que sea agradable para el jugador y que concorde con la temática del videojuego. Con esto, se pretende maximizar el objetivo de todo videojuego, el entretenimiento.