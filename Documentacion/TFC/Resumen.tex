\lettrine[lines=2,findent=2pt,nindent=3pt,loversize=0.1]{\textcolor[gray]{0.4}{E}}{l} trabajo en cuestión se trata de un videojuego de acción en tercera persona, realizado con el motor gráfico \ac{UE4}. El juego ha sido desarrollado usando el lenguaje de programación C++ y extendiendo la funcionalidad del mismo con el lenguaje de \textit{scripting} del motor gráfico, \textit{Blueprints}. El videojuego consiste en una campaña cooperativa con posibilidad de que dos jugadores la completen de forma online. Para ello el videojuego usa el subsistema online de Steam\footnote{https://store.steampowered.com}, por lo que es necesario tener abierto Steam para poder jugar el videojuego. El videojuego añadirá una mecánica consistente en la resolución de un minijuego de tipo puzzle para acceder a ciertas zonas del mapa. El minijuego se trata de una versión del clásico juego de mesa ``\textit{Klotski}''\footnote{https://es.wikipedia.org/wiki/Klotski}. En él, tendrás que mover una serie de bloques sobre una superficie plana para conseguir guiar la pieza principal hacia la salida del tablero.

El argumento del videojuego es el siguiente: ``Eres \textit{Alyssa Theras} parte de la poca resistencia que queda en el planeta Tierra. Habéis sido invadidos por una raza extraterrestre que quiere esclavizaros. Tu misión será encontrar a otros miembros de la resistencia y unir fuerzas con ellos para liberar a la Tierra del invasor enemigo.''

La historia del videojuego se divide en dos niveles. En el primer nivel, el objetivo es derrotar a los enemigos esclavos e ir recabando información que te permita saber cuál es el punto débil de los alienígenas, para poder eliminarlos. En este nivel conocerás a dos personajes que se unirán a tu grupo y te darán información para eliminar a los aliens. En el segundo nivel te infiltras en la nave alienígena enemiga. En este punto ya sabéis la forma para conseguir derrotarlos y el objetivo será abrirse paso por la nave y lograr eliminar al jefe final.