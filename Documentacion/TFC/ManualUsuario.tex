\lettrine[lines=2,findent=2pt,nindent=3pt,loversize=0.1]{\textcolor[gray]{0.4}{E}}{l} manual de usuario se va a organizar en forma de preguntas y respuestas frecuentes. Se organiza así ya que es más sencillo buscar la respuesta a una pregunta concreta en una sección del manual, y no hay que leer todo el manual completo para una duda puntual.

\subsection{¿Qué hacer si se muestra un error de STEAM al iniciar \textit{Alliance}?}

Si al iniciar \textit{Alliance} nos encontramos con un error como el de la Figura \ref{ErrorSteam}, tan sólo tendremos que iniciar sesión en una cuenta de STEAM y reiniciar \textit{Alliance}. El error se debe a que el juego para su funcionamiento online usa el subsistema online de STEAM y sin iniciar sesión en STEAM no se puede hacer uso del mismo.

\begin{figure}[H]
  \centering
  \includegraphics[width=12cm]{./images/ErrorSteam.png}
  \caption{Error de STEAM al iniciar el juego}
  \label{ErrorSteam}
\end{figure}


\subsection{¿Cómo jugar una partida de \textit{Alliance}?}

\begin{itemize}
\item En primer lugar, haz click en el ejecutable de \textit{Alliance}, para iniciar la aplicación.
\item Aparecerá el menú principal del juego, que se puede ver en la Figura \ref{MenuPpal}.

\begin{figure}[H]
  \begin{minipage}{0.5\textwidth}
    \centering
    \includegraphics[width=8cm]{./images/MenuPpal.png}
    \caption{Menú principal de \textit{Alliance}}
    \label{MenuPpal}
  \end{minipage}%
  \hspace{1mm}
  \begin{minipage}{0.5\textwidth}
    \centering
    \includegraphics[width=8cm]{./images/EleccPers.png}
    \caption{Elección de personaje}
    \label{EleccPers}
  \end{minipage}
\end{figure}

\item En este menú haremos click en <<Nueva Partida>> para iniciar una nueva partida, la cual vamos a <<\textit{hostear}>>. Esto quiere decir que actuaremos como servidor en el caso de que se una un segundo jugador a nuestra partida. 

\item En la interfaz de la Figura \ref{EleccPers}, se puede observar un \textit{zoom} sobre los personajes principales. Elegimos al personaje con el que queramos jugar, haciendo click en el botón con su nombre. Podemos reconocer nuestra selección gracias a las partículas existentes detrás del personaje elegido. 

\item Una vez estemos seguros de la elección de personaje, hacer click en <<Continuar>>. Saldrá una pantalla de carga tras la cual apareceremos en el inicio del primer nivel, y podremos jugarlo.
\end{itemize}

\pagestyle{notsection}

\subsection{¿Cómo unirme a una partida de \textit{Alliance}?}

\begin{itemize}
\item En primer lugar, haz click en el ejecutable de \textit{Alliance}, para iniciar la aplicación.
\item Aparecerá el menú principal del juego, que se puede ver en la Figura \ref{MenuPpal}.
\item En este menú haremos click en <<Unirse a Partida>> para unirnos a una partida ya existente.

\begin{figure}[H]
  \begin{minipage}{0.5\textwidth}
    \centering
    \includegraphics[width=8cm]{./images/BuscandoPartida.png}
    \caption{Buscando partida...}
    \label{BuscandoPartida}
  \end{minipage}%
  \hspace{1mm}
  \begin{minipage}{0.5\textwidth}
    \centering
    \includegraphics[width=8cm]{./images/PartidaNoEncontrada.png}
    \caption{Error. Partida no encontrada}
    \label{PartidaNoEncontrada}
  \end{minipage}
\end{figure}

\item Inmediatamente \textit{Alliance} comenzará a buscar una partida a la cual puedas unirte, como puede observarse en la Figura \ref{BuscandoPartida}. En el caso de que no haya partidas a las cuales puedas unirte, saldrá el mensaje de error que se muestra en la Figura \ref{PartidaNoEncontrada}. Puedes realizar una nueva búsqueda haciendo click sobre <<Aceptar>>.

\item Si se encuentra una partida a la cual puedas unirte aparecerá, una pantalla como la de la Figura \ref{Success}. Saldrá una cuenta atrás de diez segundos para que aceptes unirte a la partida. Si pasados los diez segundos no te has unido a la partida, deberás proceder a realizar una nueva búsqueda de partidas.

\begin{figure}[H]
  \centering
  \includegraphics[width=9cm]{./images/Success.png}
  \caption{Éxito al encontrar una partida en \textit{Alliance}}
  \label{Success}
\end{figure}

\item Una vez que se ha aceptado unirse a una partida, saldrá una pantalla de carga y tras esta, apareceremos en el nivel en el que se encuentre el jugador que \textit{hostea} la partida, controlando al segundo personaje principal no controlado por el \textit{host} de la partida.
\end{itemize}


\subsection{¿Cómo salir del juego?}

Existen varias maneras de salir del juego, dependiendo de si te encuentras jugando alguno de los niveles de \textit{Alliance}, o si te encuentras navegando por el menú principal (véase Figura \ref{MenuPpal}).

\begin{itemize}
\item \textbf{Si te encuentras navegando por el menú principal.}

Volver hacia las opciones del principio (haciendo click en <<Volver>>), que se muestran en la Figura \ref{MenuPpal}. Una vez en dichas opciones, hacer click en <<Salir>>.

\item \textbf{Si te encuentras jugando alguno de los niveles.}

\begin{itemize}
\item Pausar el juego, presionando la tecla \texttt{P} (o la tecla Escape) y hacer click en <<Salir del juego>>.
\item Dejar que los enemigos nos maten. En la interfaz que aparece (véase Figura \ref{YouLose}), hacer click en <<Volver al Menú>>. Una vez en el menú principal, escoger la opción <<Salir>>, como se ha indicado anteriormente. 
\end{itemize}
\end{itemize}

\renewcommand\bcStyleTitre[1]{\large\hspace*{1.8in}\textcolor{red!100}{#1}}
\begin{bclogo}[
  couleur=red!15,
  arrondi=0.25,
  logo=\hspace*{1in}\bctakecare,
  barre=none,
  noborder=true]{\hspace*{0.15in} Advertencia.}
\itshape \vspace*{0.15in}
\textit{Alliance} no implementa un sistema de guardado de partidas, por lo que si sales de una partida, no se guardará tu progreso. La próxima vez que inicies \textit{Alliance} deberás empezar una partida nueva, comenzando el juego desde el principio.
\end{bclogo}


\subsection{¿Cómo jugar el minijuego?}

El minijuego que se ha implementado en \textit{Alliance} como mecanismo para abrir ciertas puertas es el clásico <<Klotski>>. Este juego se compone de un tablero, sobre el cual se colocan varias fichas de distintos tamaños y en distintas orientaciones. Existe una ficha, de color rojo (generalmente), que es la que necesitamos mover hacia la salida para ganar. Los únicos movimientos permitidos son horizontales y verticales sobre cada ficha, siempre y cuando haya espacio que lo permita.

En la implementación virtual, existen fichas de colores azules y rojo. La ficha con color rojo es la ficha que hay que llevar al hueco verde para conseguir ganar el minijuego y abrir la puerta. 

Tan sólo se puede mover una ficha si se ha seleccionado. Sabemos si una ficha está seleccionada porque su color cambia a rosa, como se puede ver en la Figura \ref{Minijuego}, y la selección de fichas se realiza pulsando la combinación de teclas \texttt{SHIFT+E} (para seleccionar la siguiente pieza) y \texttt{SHIFT+Q} (para seleccionar la pieza anterior).

Para mover una ficha seleccionada, se usan las teclas \texttt{W} (para mover la ficha hacia arriba), \texttt{S} (para moverla hacia abajo), \texttt{A} (para moverla hacia la izquierda) y \texttt{D} (para moverla hacia la derecha).

\begin{figure}[H]
  \centering
  \includegraphics[width=12cm]{./images/Minijuego.png}
  \caption{Ejecución del minijuego durante una partida de \textit{Alliance}}
  \label{Minijuego}
\end{figure}

Durante la ejecución del minijuego, el jugador que lo está jugando no puede moverse ni atacar.


\subsection{¿Cómo incrementar la vida y la estamina?}

La vida y la estamina son los dos rasgos más importantes que posee el jugador. La vida es necesaria para no perder la partida y la estamina es necesaria para poder realizar ataques y para poder correr más rápido. Por esto, se han implementado dos métodos para poder aumentar dichas cualidades.

\begin{itemize}
\item Tanto la vida, como la estamina, se regeneran con el tiempo. Este porcentaje de regeneración es muy bajo, aunque en situaciones de calma en las que no haya combate es muy útil para comenzar los siguientes combates bien regenerado.

\item Existen cajas, barriles y otros destructibles (véase la Figura \ref{Breakables}). Al romper estos objetos, existe la posibilidad de que aparezcan \textit{pickups} (véase la Figura \ref{Pickup}), que al cogerlos proporcionan vida y estamina.

\begin{figure}[H]
  \begin{minipage}{0.33\textwidth}
    \centering
    \includegraphics[width=4cm]{./images/Dest_Barr2.png}
  \end{minipage}%
  \hspace{0.5mm}
  \begin{minipage}{0.33\textwidth}
    \centering
    \includegraphics[width=4cm]{./images/Dest_Barr.png}
  \end{minipage}
  \hspace{0.5mm}
  \begin{minipage}{0.33\textwidth}
    \centering
    \includegraphics[width=4cm]{./images/Dest_Box.png}
  \end{minipage}
  \caption{Distintos tipos de breakables}
  \label{Breakables}
\end{figure}

Coger un \textit{pickup} de vida te aumenta en 15 puntos tu salud, mientras que coger un \textit{pickup} de estamina aumenta en 25 puntos tu estamina.


\begin{figure}[H]
  \begin{minipage}{0.5\textwidth}
    \centering
    \includegraphics[width=5cm]{./images/Health.png}
  \end{minipage}%
  \hspace{1mm}
  \begin{minipage}{0.5\textwidth}
    \centering
    \includegraphics[width=5cm]{./images/Stamina.png}
  \end{minipage}
  \caption{Pickups de salud y estamina}
  \label{Pickup}
\end{figure}
\end{itemize}

\subsection{¿Qué tipos de enemigos existen?}

En \textit{Alliance} hay tres tipos de enemigos principales, además de tres jefes enemigos.

\begin{itemize}
\item Los enemigos principales son:
\begin{itemize}
\item Enemigos a melé (Figura \ref{Mele}). La forma de ataque de estos enemigos es acercarse al jugador más cercano cuando le ven y una vez que están cerca, atacarle. Sus ataques quitan \textbf{8} puntos de salud a los personajes controlados por humanos. Tienen una vida total de \textbf{30} puntos de salud.
\item Enemigos a distancia (Figura \ref{Distancia}). Estos enemigos se mantienen a una cierta distancia de los jugadores, desde la que disparan proyectiles en forma de lanza a los mismos. Sus  ataques quitan \textbf{5} puntos de salud a los personajes controlados por humanos. Tienen una vida total de \textbf{20} puntos de salud.
\item Enemigos tanque (Figura \ref{Tanque}). Son enemigos de movimiento y ataque lento, pero muy poderoso. Además pueden recibir mucho daño antes de morir. Su forma de ataque es igual que la de los enemigos a melé, acercándose hacia el jugador más cercano y atacándole. Sus ataques quitan \textbf{10} puntos de salud a los personajes controlados por humanos. Tienen una vida total de \textbf{80} puntos de salud.
\end{itemize}

\begin{figure}[H]
  \begin{minipage}{0.33\textwidth}
    \centering
    \includegraphics[width=5cm]{./images/guerrero.jpg}
    \caption{Enemigos a melé}
    \label{Mele}
  \end{minipage}%
  \hspace{0.5mm}
  \begin{minipage}{0.33\textwidth}
    \centering
    \includegraphics[width=3cm]{./images/distancia.jpg}
    \caption{Enemigos a distancia}
    \label{Distancia}
  \end{minipage}
  \hspace{0.5mm}
  \begin{minipage}{0.33\textwidth}
    \centering
    \includegraphics[width=5cm]{./images/tanque.jpg}
    \caption{Enemigos tanque}
    \label{Tanque}
  \end{minipage}
\end{figure}

\item Los jefes enemigos son:
\begin{itemize}
\item \textbf{Henkka} (Figura \ref{Henkka}). Es el jefe de los esclavos de la primera zona. Ataca cuerpo a cuerpo con dos grandes mazas. Sus ataques quitan \textbf{15} puntos de salud a los personajes controlados por humanos. Tiene una vida total de \textbf{1000} puntos de salud.
\item \textbf{Euronymous} (Figura \ref{Euronymous}). Es el jefe de los esclavos de la segunda zona. Ataca cuerpo a cuerpo con una espada, su movimiento es más rápido. Sus ataques quitan \textbf{20} puntos de salud a los personajes controlados por humanos. Tiene una vida total de \textbf{800} puntos de salud.
\item \textbf{Shiva} (Figura \ref{Shiva}). Es el jefe de los alienígenas enemigos. Sus ataques los lleva a cabo con sus dos grandes puños. Estos ataques quitan \textbf{25} puntos de salud a los personajes controlados por humanos. Tiene una vida total de \textbf{1200} puntos de salud.
\end{itemize}

\begin{figure}[H]
  \begin{minipage}{0.33\textwidth}
    \centering
    \includegraphics[width=5cm]{./images/henkka.jpg}
    \caption{Boss Henkka}
    \label{Henkka}
  \end{minipage}%
  \hspace{0.5mm}
  \begin{minipage}{0.33\textwidth}
    \centering
    \includegraphics[width=5cm]{./images/euronymous.jpg}
    \caption{Boss Euronymous}
    \label{Euronymous}
  \end{minipage}
  \hspace{0.5mm}
  \begin{minipage}{0.33\textwidth}
    \centering
    \includegraphics[width=5cm]{./images/shiva.jpg}
    \caption{Boss Shiva}
    \label{Shiva}
  \end{minipage}
\end{figure}

\begin{greenbox}[Sobre el ataque de los enemigos...]
\begin{center}
Todos los ataques de los enemigos quitan una cantidad de puntos de salud a los personajes controlados por humanos. Los personajes controlados por la \ac{IA} no pueden morir, tomen el daño que tomen.
\end{center}
\end{greenbox}
\end{itemize}

\subsection{¿Cómo puedo atacar con \textit{Alyssa}?}

\textit{Alyssa} posee varios ataques, que pueden ejecutarse de las siguientes formas:

\begin{itemize}
\item Ataque básico. En este ataque, \textit{Alyssa} producirá daño golpeando con su maza. Se puede llevar a cabo pulsando el \textbf{botón izquierdo} del ratón. Se pueden encadenar varios de estos ataques en forma de combo, pulsando de forma seguida en repetidas ocasiones este botón.
\item Ataque especial. Si se pulsa el \textbf{botón derecho} del ratón, \textit{Alyssa} lanzará una bola de fuego. Esta bola hace mucho daño a los enemigos, pero también consume mucha estamina.
\item Ataque mientras bloquea. Si mientras se está bloqueando (con la tecla \texttt{F}), se pulsa el \textbf{botón derecho} del ratón, se producirá un daño en área a los enemigos, empujándolos hacia fuera de dicho área. Este ataque es muy útil si nos encontramos rodeados de enemigos.
\item Ataque en carrera. Si se pulsa el \textbf{botón izquierdo} del ratón mientras se está \textit{sprintando} (con la tecla \texttt{SHIFT} pulsada), se producirá un ataque en salto, que es más poderoso que el ataque básico pero a su vez consume mayor estamina.
\end{itemize}

\subsection{¿Cómo puedo atacar con \textit{Morten}?}

\textit{Morten} posee también varios ataques, como son:

\begin{itemize}
\item Ataque básico. En este ataque, \textit{Morten} producirá daño disparando con sus pistolas. Se puede llevar a cabo pulsando el \textbf{botón izquierdo} del ratón. El daño que produce este ataque no es muy alto, pero se ve compensado con la rapidez en la ejecución (lo que permite atacar más veces en el mismo tiempo) y con el poco gasto de estamina.
\item Ataque especial. Si se pulsa el \textbf{botón derecho} del ratón, \textit{Morten} lanzará una granada. Dicha granada explotará pasado un pequeño tiempo, produciendo un gran daño en área a los enemigos que se encuentren dentro del rango de explosión.
\end{itemize}

\begin{greenbox}[Tanto \textit{Alyssa} como \textit{Morten}]
\begin{center}
pueden bloquear ataques pulsando la tecla \texttt{F}, \textit{sprintar} mientras se mueven si se mantiene pulsada la tecla \texttt{SHIFT} y realizar esquivas pulsando el espacio.
\end{center}
\end{greenbox}
\end{itemize}

\subsection{¿Cómo acceder a las partes del mapa bloqueadas?}

En el mapa de \textit{Alliance} existen ciertas partes no accesibles, bloqueadas por un muro invisible como el de la Figura \ref{Wall}. El motivo de este muro es prevenir pasar a algunas zonas sin completar objetivos en las zonas anteriores.

\begin{figure}[H]
  \centering
  \includegraphics[width=12cm]{./images/InvisibleWall.png}
  \caption{Muro invisible, que evita el paso a ciertas partes del mapa}
  \label{Wall}
\end{figure}

Para poder acceder a las siguientes zonas del mapa, será necesario haber acabado por completo con todos los enemigos de las zonas anteriores. Habremos completado este objetivo si el muro desaparece, permitiendo el paso a la siguiente zona.

\subsection{¿Cómo interactuar con los diálogos?}

Durante el juego aparecerán diálogos, que explicarán la historia del mundo, te guiarán por el camino y te ayudarán con los controles y mecánicas del juego. Estos diálogos tienen la apariencia que se puede observar en la Figura \ref{Dialogo}.

Mientras estás leyendo el texto de los diálogos, el juego estará pausado. Para seguir leyendo el texto, se hará click en la flecha que aparece abajo a la derecha. Cuando no quede más diálogo por leer, este desaparecerá automáticamente y volverás a controlar al personaje principal.

\renewcommand\bcStyleTitre[1]{\large\hspace*{1.8in}\textcolor{red!100}{#1}}
\begin{bclogo}[
  couleur=red!15,
  arrondi=0.25,
  logo=\hspace*{1in}\bctakecare,
  barre=none,
  noborder=true]{\hspace*{0.15in} Advertencia.}
\itshape \vspace*{0.15in}
Cabe destacar que una vez que se ha leído un diálogo, no se puede volver a leer, por lo que ten cuidado al saltarlo y obviar cierta información.
\end{bclogo}

\begin{figure}[H]
  \centering
  \includegraphics[width=12cm]{./images/Dialogo.png}
  \caption{Interfaz de diálogos de \textit{Alliance}}
  \label{Dialogo}
\end{figure}


\subsection{¿Cómo cambiar el personaje del juego?}

Mientras se está jugando no existe la posibilidad de cambiar el personaje que es controlado por el jugador y continuar con el juego. Existen dos formas de cambiar el personaje, pero ambas necesitan empezar el juego de nuevo:

\begin{itemize}
\item Dejar que los enemigos nos maten. En la pantalla que sale cuando el jugador pierde (se puede ver en la Figura \ref{YouLose}), hacer click en <<Volver al Menú>>. Una vez en el menú, se puede iniciar una nueva partida escogiendo al personaje que se quiera, entre los dos disponibles.

\begin{figure}[H]
  \centering
  \includegraphics[width=10cm]{./images/YouLose.png}
  \caption{Interfaz gráfica cuando el jugador pierde}
  \label{YouLose}
\end{figure}

\item Pausar el juego, presionando la tecla \texttt{P} y hacer click en <<Salir del juego>>. Reiniciar \textit{Alliance} e iniciar una nueva partida, escogiendo al personaje deseado entre los dos personajes disponibles.
\end{itemize}

\renewcommand\bcStyleTitre[1]{\large\hspace*{1.8in}\textcolor{red!100}{#1}}
\begin{bclogo}[
  couleur=red!15,
  arrondi=0.25,
  logo=\hspace*{1in}\bctakecare,
  barre=none,
  noborder=true]{\hspace*{0.15in} Advertencia.}
\itshape \vspace*{0.15in}
\textit{Alliance} no implementa un sistema de guardado de partidas, por lo que si sales de una partida para cambiar tu personaje y empezar una partida nueva, no se guardará tu progreso y tendrás que empezar desde el principio.
\end{bclogo}


\subsection{¿Cómo ver los créditos de \textit{Alliance}?}
\begin{itemize}
\item En primer lugar, haz click en el ejecutable de \textit{Alliance}, para iniciar la aplicación.
\item Aparecerá el menú principal del juego, que se puede ver en la Figura \ref{MenuPpal}.
\item En este menú haremos click en <<Creditos>> para visualizar los créditos del videojuego.

\begin{figure}[H]
  \centering
  \includegraphics[width=12cm]{./images/Creditos.png}
  \caption{Créditos del juego}
  \label{Creditos}
\end{figure}

\item Se mostrará una interfaz como la de la Figura \ref{Creditos}, en la cual se mostrarán los créditos de \textit{Alliance} con una animación de abajo a arriba. Cuando se hallan mostrado todos los créditos, se activará la opción de volver al menú principal del juego.
\end{itemize}
