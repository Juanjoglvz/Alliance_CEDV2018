El proyecto al final ha resultado ocuparnos más tiempo de lo que habíamos estimado en un principio. Se han finalizado todas las partes importantes que se habían planteado en un principio, pero muchas ideas originales e interesantes que surgían una vez ya estaba empezado el desarrollo (Como la inclusión de un hechizo de curación). Este tipo de cosas no estaban planificadas en un principio, y su inclusión no supone la no realización de los objetivos planteados en la propuesta inicial.
\\

Por otra parte, el género en el que se ha planteado el proyecto se presta más a juegos largos, con más componentes de progresión de los que presentamos en esta solución. Es por eso que el juego se ha planteado más como una forma de \textbf{demostración}. Esto significa que, aunque los elementos principales del juego estén presentes, no se ha pensado en este juego como un producto completo y listo para la venta al público. Aún así, algunos elementos, como es la historia o la eliminación de la progresión, se han adaptado para tratar que parezca un producto un poco más completo, aunque siempre teniendo en cuenta que el juego sería una \textit{demo} de un juego mucho más grande y complejo.
\\

También queremos aprovechar para volver a recalcar lo limitados que nos hemos sentido a la hora de, por ejemplo, trabajar con animaciones preestablecidas. Muchas de las posibilidades en el sistema de combate tuvieron que ser redefinidas cuando empezamos a ver las posibilidades que teníamos a la hora de trabajar con ellas (las animaciones). También nos hemos sentido frustrados frente a la falta de conocimientos que teníamos en ciertos temas como, por ejemplo y sin querer ser redundantes, al trabajar con animaciones. No sabemos si es porque esta parte no es realmente algo que concierne a los programadores(Suponemos que el \textbf{realizar} las animaciones como tal no lo son), pero sí que nos faltaba mucha práctica a la hora de trabajar con ellas, manipularlas y montar todo el sistema de comunicación entre Blueprints que hemos hecho en nuestro proyecto.
\\

Por otra parte, también nos hemos sentido, bien frustrados, bien descontentos, con la inteligencia artificial. En general, nuestra idea de lo que queríamos de inteligencia artificial para los enemigos del juego era más compleja. Al final, quizás por falta de tiempo, o por falta de experiencia o conocimientos, hemos acabado desarrollando una inteligencia artificial más genérica, válida para casi todos los tipos de enemigos con ligeras modificaciones, aunque esa no fuera la idea inicial.
\\

Como propuesta de trabajo futuro, volvemos a reiterar que la idea general del juego es la de una \textbf{demo}. La propuesta de trabajo más inmediata es la de expandir el juego, planteando nuevos mapas y alargando la historia, quizás abandonando la Tierra para perseguir a sus enemigos. Para expandir las mecánicas se puede añadir un sistema de progresión, para desbloquear nuevas habilidades y mejoras (O incluso podemos empezar con algunas de las habilidades especiales de la demo bloqueadas). Otro de los elementos que se deberían mejorar en corto-medio plazo es la inteligencia artificial de los enemigos. No sólo volverla más compleja, si no también evitar ciertos comportamientos poco inteligentes descritos anteriormente. También podrían añadirse nuevos tipos de minijuego, para que no sea siempre el mismo tipo de puzzle.
\\

Algunas posibles mejoras al juego cuando contamos con la ampliación del juego es la inclusión de un modo multijugador para más personas (No sólo dos). Esto supondría que, los aliados que reclutamos en el proyecto que presentamos (Y que recordemos que se trataría como una \textit{demo}), continuarían con nosotros el resto del juego. Además esto supondría también un esfuerzo extra a la hora de diseñar estos personajes, que ahora mismo son tratados como acompañantes que nunca van a ser jugables, y por tanto necesitaríamos hacer más complejo su sistema de combate. También significaría hacer un menú de elección de personaje más complejo, y que aparezca a la hora de unirse a partidas, y diseñar un menú de búsqueda de partidas más adaptado.